\documentclass[letterpaper,10pt,onecolumn]{article}
\usepackage[utf8]{inputenc}
\usepackage{geometry}
\geometry{
    letterpaper,
    total={8.5in,11in},
    left=1.0in,
    right=1.0in,
    top=1.0in,
    bottom=1.0in,
}

\title{Computer Networks Project Proposal (TBD)}

\author{Mehrdad Khani Shirkoohi, Shichao Yue, Songtao He \\
\{khani, scyue, songtao\}@mit.edu
}

\begin{document}
\date{}
\maketitle
\section{Introduction (Research Problem)}
\paragraph*{}Internet plays significant role in our todays world. Almost all internet consumers and many of their favorite companies providing application services care significantly about the packet delay they experience in the networks. A major portion of these network delays are due to existence of queues in routers which tend to move the networks toward more utilization of bandwidth. So, on one hand it is desirable to reduce the delays and on the other hand, a better utilization of our bandwidth is a goal. Hence, precise management of our queues is of great importance these days.

Efficient queue management has been always challenging in our router buffers. One solution, which has been deployed for several years, is to provide as much buffer as it is possible in order to avoid dropping any packets. This solution is not considered a perfect one any more these days, since it leads to a problem known as bufferbloat[citation needed here]. Bufferbloat is experiencing high latency due to exess buffering of packets in our packet-network switches. The bufferbloat phenomenon was initially described as far back as in 1985.[1] It gained more widespread attention starting in 2009.[2] 

In this project, we are going to focus on the packet dropping rate policy taking Reinforcement Learning approach.



\section{Research Methodology}
\section{Schedule and Plan}
\section{Resources}

\end{document}

